{\scshape\Large\bfseries\centering ABSTRACTION OF THESIS \par}
\vspace{1cm}
\begin{flushleft}
{\justify
Nowadays, The data throughout the world has been exponentially increasing and become extremely valuable. So it is challenging for us to manage and exploit it. Although there are many advanced database management systems which have been developed to solve this challenge, they have still exposed their drawbacks in managing the  big volume data sets. Their default executor showed bad performance in executing the query on these data sets. Hence, it is neccessary for us to find the solutions to improve the performance of database management systems.  \par}
\vspace{0.5cm}
{\justify
There are many factors that affect the performance of database management systems, such as: cost model, cardinality estimation, etc. It is hard to focus on all of these factors to solve the problem. Therefore, Finding the factor that has the most significant effect to  the performance of database management systems plays an important role.  \par }
\vspace{0.5cm}
{\justify
This project focus on finding this main factor and proving its effect by benchmarking it in PostgreSQL and analyzing its effect. Besides, I also propose the published solutions to reduce its effect.\par}
\vspace{0.5cm}
{\justify
First of all, I introduce the context about the way that a query is executed. I will focus on the steps that a query is processed. And after that, I find and prove the step that affects to the query performance the most. And then i dig deeper into this step to find the factors that affect to query performance by researching query optimization. \par}
\vspace{0.5cm}
{\justify
Before benchmarking the factors, I find the benchmarks that are used to benchmark executing query performance of database management systems. And then i try and research them to find the best benchmarks for benchmarking the factor in PostgreSQL. \par }
\vspace{0.5cm}
{\justify
Next, I benchmark the factor in PostgreSQL by using the benchmarks that i already choose. And then i analyze the results to show the effects of the factor to query performance. And finally, I will propose some solutions that have been published to reduce their effects and improve the query performance. \par }
\vspace{0.5cm}
{\justify
This project is made at HADAS team, LIG Laboratory under the supervisions of Dr. Christophe Bobineau - Associate Professor of Grenoble Institute of Technology, Maxence Alouche - PhD student at Grenoble INP and Dr. Vu Tuyet Trinh - SOICT.
\par}
\end{flushleft}