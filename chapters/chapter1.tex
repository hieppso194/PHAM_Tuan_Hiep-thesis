%this is chapter about the process of query execution.

\chapter{Introduction}\label{chapter:introduction}

\section{Context and objective of the thesis}

\subsection{Context}

Nowadays, new information sources, such as social networks, mobile devices,
sensors, recreationals generate unprecedented volumes of data. This data is
stored in datasets that can reach petabytes or exabytes in size. 

It is undoubted that these big volumes of data contain a significant value to our
life. For example, desision making carried out by management teams and the creation
of new business can be derived by these large amounts of information. However, to exploit
these big volumes of data is not an easy task and there are many challenges ahead to achieve their great value. One of these challenges is the query performance of data management systems on big data sets.

Current data management systems have exposed their drawbacks, especially in query performance. When executing the queries with these DBMSs, there are long-running queries, which not only consumes system resources that lead the server and application to run slowly, but may also lead to table locking. So, query optimization becomes an important task.

There are many factors that affect to query optimization and it is hard to focus
on all of them to improve the query performance. I identified
that cardinality estimation has a big impact on query performance, and investigate this issue in particular.

\subsection{Objective of my thesis}

In this thesis, I have decided to focus primarily on cardinality estimation and
showing its effect on query performance.  Besides, I will also propose some
published solutions to reduce its effects.  There are steps in my thesis:

\begin{itemize}
    \item Understanding how a query is executed.
    \item Identifying the factors in query optimization.
    \item Finding the best benchmarks for cardinality estimation.
    \item Benchmarking and analyzing the effect of cardinality estimation on
        query performance;
    \item Proposing published solutions.
\end{itemize}
